\section{Question 2}

L'aximon d'homogénétié stipude qu'un accroissement d'un risque paar une facteur $c$ devrait impacter la mesure de rsique d'un facteur $c$ équivalent. Formellement, on a:\\
soit une mesure de risque rho, une variable aléatoire de perte L et un facteur c,\\
$\rho(cL) = c*\rho(L)$\\

Intuitivement, si le risque est en denomination quelconque, on s'attendrait à ce que ce risque soit de magnetude relativement identique dans une autre devise. Autrement dit, on ne s'attends pas à ce qu'un changement de devise impact le niveau de danger du risque. \\

La principale critique de cet axiome est qu'un risque plus grand d'un facteur $c$ peut engendrer d'autres risques associés à un accroissement du risque $L$. Par exemple, on pourrait penser que si le risque était multiplié par une facteur 1 000 000 dans la même devise, qu'il soit possible qu'il inclue maintenant un risque de liquidité. Dans un tel cas on aurait que: \\
$\rho(cL) > c*\rho(L)$\\