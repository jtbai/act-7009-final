\section{Question 3}

Soit les trois mesures de risque suivantes :\\
$\rho^a= sup\{L(w_i) : i= 1,2,...,m\}$\\
$\rho^b=  sup\{L(w_i) : i= 1,2,...,m\}$\\
$\rho^c= sup\{\frac{L(w_i)+L(w_j)}{2} : i \neq j \in \{1,2,...,m\}\}$\\


\textbf{a. preuve de l'homogénéité des mesures $\rho^a, \rho^b, \rho^c$}\\
\textbf{$\rho^a$}\\
$sup\{L(cw_i)\} = sup\{cL(w_i)\}$\\
$ = c*sup\{L(w_i)\}$\\

\textbf{$\rho^b$}\\
$sup\{L(cw_i)\} = sup\{cL(w_i)\}$\\
$ = c*sup\{L(w_i)\}$\\

\textbf{$\rho^c$}\\
$sup\{\frac{L(cw_i)+L(cw_j)}{2}\} = sup\{\frac{cL(w_i)+cL(w_j)}{2}\}$\\
$ = sup\{\frac{2*c}{2} \frac{L(w_i)+L(w_j)}{2}\}$\\
$ = c*sup\{ \frac{L(w_i)+L(w_j)}{2}\}$\\

\textbf{b. preuve de monotonicite des mesures $\rho^a, \rho^b, \rho^c$}\\
???\\

\textbf{c. preuve d'invariance a la translation des mesures $\rho^a, \rho^b, \rho^c$}\\

\textbf{$\rho^a$}\\
$sup\{L(w_i+a)\} = sup\{L(w_i)+a\}$\\
$ = sup\{L(w_i)\} + a$\\

\textbf{$\rho^b$}\\
$sup\{L(w_i+a)\} = sup\{L(w_i)+a\}$\\
$ = sup\{L(w_i)\}+a$\\

\textbf{$\rho^c$}\\
$sup\{\frac{L(w_i+a)+L(w_j+a)}{2}\} = sup\{\frac{L(w_i)+a+L(w_j)+a}{2}\}$\\
$ = sup\{\frac{2*a}{2} + \frac{L(w_i)+L(w_j)}{2}\}$\\
$ = sup\{ \frac{L(w_i)+L(w_j)}{2}\} + a$\\

\textbf{d. preuve de sous-additivite des mesures $\rho^a, \rho^b, \rho^c$}\\
???\\

\textbf{e. interpretation des mesures $\rho^a, \rho^b, \rho^c$}\\
$\rho^a$ : maximum des pertes\\
$\rho^b$ : $var_{99}$ des pertes\\
$\rho^c$ : $cte_{98}$ des pertes\\

