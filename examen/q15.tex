\section{Question 15}
a. $X \leq_{SL} Y$ implique que $TVaR(x) \leq TVaR(Y)$\\

b. Voir p.282 de l'article\\

c. $\sigma_2 \geq \sigma_1$, alors la loi normale ayant une variance $\sigma_2$ aura une queue plus epaisse et une probabilte modale moins probable. Ainsi, le risque utilisant $\sigma_2$ aura plus de probabilite sur les valeurs plus grande de X.\\

d. Soit deux risque $X_1 et X_2$ pour lesquel un evenement arrive. Le cout de cet evenement est tel que $X_1$ ne depasse pas la prime stop loss, tandis  que $X_2$ la depasse. Aussi on a que la probabilit/ que $X_1$ et $X_2$ aie ete plus grand est la meme. Ainsi si on additionne tous les couts depassant les valeurs de $X_1$ et $X_2$, clairement, $X_2$ sera plus grand car il additionne deja le maximum de la perte (dans la section stop loss) tandis que $X_1$ aura encore quelques valeurs avant de se rendre a cette perte maximale (dans la zone stop loss).\\