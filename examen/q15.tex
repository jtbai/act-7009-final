\section{Question 15}
a. $X \leq_{SL} Y \leftrightarrow TVaR(x) \leq TVaR(Y)$\\

b. Copie de la page p.582 de l'article\\

Alors, en choissisant d = $VaR_p$, nous trouvons\\

$TVaR_p(X) = VaR_p(X) + \frac{1}{1-p}E[(X-VaR_p(X))_+]$ \\
$= \frac{f(VaR_p(X))}{1-p}$ \\
$\leq \frac{f(VaR_p(Y))}{1-p}$ \\
$= VaR_p(Y) + \frac{1}{1-p}E[(Y-VaR_p(Y))_+]$ \\
$\leq TVaR_p(Y).$\\


Pour prouver l'autre implication, on assume que les TVaR sont ordonnées pour tout $p \in (0,1)$. Notez que pour toute variable aléatoire X, nous avons que\\

$E[(X-d)_+] = (F^{-1}_X(U) - d)_+ $\\
$ = \int_{F_X(D)}^{1} Q_q[X]dq-d(1-F_x(d))$.\\


Alors, pour d tel que $ 0 \le F_x(d) \le 1$, nous trouvons

$E[(X-d)_+] = (TVaR_{F_X(d)}(X) - d)(1-F_X(d)) $\\
$\leq (TVaR_{F_X(d)}(Y) - d)(1-F_X(d)) $\\
$= \int_{F_X(D)}^{1} VaR_q(Y) dq - d(1-F_X(d))$\\
$ = \int_{F_Y(D)}^{1} VaR_q(Y) dq - d(1-F_Y(d)) +
\int_{F_X(D)}^{F_Y(D)} VaR_q(Y) dq - d(F_Y(d) - F_X(d))$ \\
$= E[(Y-d)_+] + \int_{F_X(D)}^{F_Y(D)} (VaR_q(Y)-q) dq$\\


En utilisant l'équivalence $q \le F_y(d) \leftrightarrow d \geq Q_q[Y]$, il est trivial de prouver que \\

$\int_{F_X(D)}^{F_Y(D)} (Q_q[Y] - d) dq \le 0$\\


c. $\sigma_2 \geq \sigma_1$, alors la loi normale ayant une variance $\sigma_2$ aura une queue plus epaisse et une probabilte modale moins probable. Ainsi, le risque utilisant $\sigma_2$ aura plus de probabilite sur les valeurs plus grande de X.\\

d. Soit deux risque $X_1 et X_2$ pour lesquel un evenement arrive. Le cout de cet evenement est tel que $X_1$ ne depasse pas la prime stop loss, tandis  que $X_2$ la depasse. Aussi on a que la probabilité que $X_1$ et $X_2$ aie ete plus grand est la meme. Ainsi si on additionne tous les couts depassant les valeurs de $X_1$ et $X_2$, clairement, $X_2$ sera plus grand car il additionne deja le maximum de la perte (dans la section stop loss) tandis que $X_1$ aura encore quelques valeurs avant de se rendre a cette perte maximale (dans la zone stop loss).\\