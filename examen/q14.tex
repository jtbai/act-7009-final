\section{Question 14}

a. Calculer $Pr(S = 3X_1+10X_2 > 50)$. En utilisant la prime d'Euler\\
$F_S(x) = \frac{\beta_1}{\beta_1-\beta_2}(1-e^{-\beta_2*x}) +  \frac{\beta_2}{\beta_2-\beta_1}(1-e^{-\beta_1*x})$\\
$F_S(50) = \frac{\frac{1}{3}}{\frac{1}{3}-\frac{1}{10}}(1-e^{-\frac{1}{10}*50}) +  \frac{\frac{1}{10}}{\frac{1}{10}-\frac{1}{3}}(1-e^{-\frac{1}{3}*50})$\\
$F_S(50) = 1.418946 + -0.4285714 = 0.9903746$\\

b. Evaluer aux points suivants\\
S = 12 + 10 = 22 (dans la region d'acceptation)\\
S = 15 + 35 = 50 (sur la ligne d'acceptation)\\
S = 15 + 40 = 55 (Hors de la zone d'acceptation)\\

c. ???
